\section{Architecture Design}
\label{sec:arch}

\tresumo{Arquitetura e Processo: Apresentação da topologia e do fluxo automatizado de deploy do cluster (Terraform -> Cloud-Init -> Ansible).}

\bnote{ATENÇÃO: As referências para as figuras fig:prov-terraform, fig:prov-os e fig:prov-cluster constam no texto mas as imagens não estão no arquivo latex inseridas com \texttt{\textbackslash begin\{figure\}}.}

\begin{figure}[htbp]
    \centering
    \includegraphics[width=0.9\linewidth]{content/assets/topology.png}
    \caption{Topology of Cloud Config on OCI.}
     \label{fig:topology}
\end{figure}

Figure~\ref{fig:prov-terraform} depicts the Terraform execution
perspective of the provisioning workflow. It covers RSA key-pair
generation, network resource creation (VCN, Internet Gateway, DRG,
Route Table, Security List), IPSec VPN configuration, compute
instance provisioning with cloud-init \texttt{user\_data}, and
the upload of Ansible playbooks and cluster assets to the master
node via SSH file provisioner.

Figure~\ref{fig:prov-os} details the OS-level configuration phase,
triggered by cloud-init on all VMs immediately after boot. It
covers the concurrent execution on the master (Python/Ansible
installation, SSH key injection, Ansible inventory setup) and
workers (SELinux and firewall disabling, chrony), followed by
the full \texttt{site.yml} Ansible playbook: base package
installation, PostgreSQL initialisation, Ambari Server setup,
ODP repository configuration, and Ambari Agent deployment across
all cluster nodes.

Figure~\ref{fig:prov-cluster} shows the cluster deployment and
recovery flow, orchestrated by Ansible calling the Ambari REST
API (\texttt{cluster\_deploy.yml} and \texttt{deploy\_tasks.yml}).
The diagram covers the happy-path deployment (VDF registration,
blueprint upload, cluster creation, and progress polling) and the
automated recovery logic supporting up to three sequential retry
attempts, each preceded by HDFS corrections and Hive database
reconfiguration, with a final fallback to manual service
initialisation (\texttt{manual\_service\_init.sh}).