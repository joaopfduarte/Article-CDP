\newpage

\section{Conclusion}
\label{sec:conclusion}

\tresumo{Conclusão: Síntese dos resultados, validação das hipóteses levantadas e trabalhos futuros.}

\bnote{Retomar os três objetivos elencados na introdução e discutir de forma concisa como cada um foi solucionado.}

\bnote{Validar as hipóteses estruturadas no Capítulo 5: 
A arquitetura ARM aguentou o NiFi? (Hipótese 1). 
Ocorreram erros de OOM no Spark/Hive? (Hipótese 2). 
A plataforma provou-se viável a custo zero no OCI Free Tier? (Hipótese 3). Resuma os acertos e as eventuais limitações técnicas encontradas.}
\bnote{Apontar também a viabilidade dessa infraestrutura para fins de ensino e pesquisa e aplicação em pequenas
organizações.}

This report presented the design and implementation of an open-source, fully automated infrastructure-as-code (IaC) framework for deploying a Data Lake. By leveraging tools such as Terraform, cloud-init, Ansible, and Apache Ambari, the proposed solution successfully orchestrates the provisioning of complex big data components---spanning ingestion, storage, processing, and governance layers---thus fulfilling the project's requirement to provide a seamless, reproducible deployment process.

The development of this framework demonstrated the feasibility of bringing enterprise-grade data management capabilities to environments constrained by tight budgets and limited operational resources. The architecture was specifically tailored to be deployed on the Oracle Cloud Infrastructure (OCI) Free Tier, utilizing ARM-based compute instances to achieve a zero-cost baseline. This approach successfully mitigated the financial and administrative barriers commonly associated with big data initiatives, validating the core design requirements.

Consequently, this automated deployment model proves highly viable for small businesses, enabling them to establish a robust analytical foundation without incurring prohibitive cloud infrastructure costs. The capability to automatically provision the necessary storage and compute layers allows smaller organizations to focus directly on data engineering and analytics rather than complex infrastructure management.

Furthermore, the framework's modular nature and reproducible deployment profiles (ranging from Default to Data Science and Software Engineering setups) make it an excellent platform to be applied in research and educational scenarios. Academic institutions and researchers can rapidly spin up disposable, fully functional clusters to conduct experiments, teach distributed system concepts, and prototype modern data pipelines in a realistic environment. Overall, the creation of this framework establishes that, through orchestrated automation and the efficient use of freely available cloud resources, the complexity of Data Lake administration can be drastically reduced, ultimately democratizing access to advanced data architectures.
